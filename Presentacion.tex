\documentclass{beamer}

\usetheme{Madrid}
\usecolortheme{default}

\usepackage[utf8]{inputenc}
\usepackage[T1]{fontenc}
\usepackage[spanish]{babel}
\usepackage{siunitx}
\usepackage{graphicx}
\usepackage{tikz}
\usetikzlibrary{arrows.meta,positioning}

\title[Arenero inteligente IoT]{Proyecto IoT: Arenero inteligente con acceso selectivo}
\author[Grupo 4]{Luis Izaguirre \and Harold Canto \and Joel Jiménez \and Hans Mendoza}
\institute[UTEC]{CS5055 -- Internet of Things}
\date{Semana 13 -- Perfil de proyecto}

\begin{document}

%------------------------------------------------
\begin{frame}
    \titlepage
\end{frame}

%------------------------------------------------
\begin{frame}{Agenda}
    \tableofcontents
\end{frame}

%------------------------------------------------
\section{Introducción y problemática}

\begin{frame}{Motivación del proyecto}
    \begin{itemize}
        \item Hogares con varios gatos $\Rightarrow$ compartir arenero genera:
        \begin{itemize}
            \item Estrés y conflictos de territorialidad.
            \item Dificultad para monitorear la salud de cada gato.
            \item Problemas de higiene y limpieza.
        \end{itemize}
        \item Areneros automáticos comerciales:
        \begin{itemize}
            \item Se enfocan en autolimpieza o monitoreo básico.
            \item No controlan qué gato accede a qué arenero.
        \end{itemize}
        \item Oportunidad: usar IoT para controlar el acceso a recursos de forma individual.
    \end{itemize}
\end{frame}

%------------------------------------------------
\begin{frame}{Problema técnico}
    \textbf{¿Cómo permitir que solo un gato autorizado use un arenero específico?}
    \vspace{0.3cm}
    \begin{itemize}
        \item Identificar de forma fiable al gato mediante un \textbf{collar BLE}.
        \item Detectar su \textbf{proximidad} al arenero a partir del \textbf{RSSI}.
        \item Accionar una \textbf{compuerta motorizada} que:
        \begin{itemize}
            \item Se abra solo cuando el gato autorizado está cerca.
            \item Se cierre de forma segura, evitando atrapamientos.
        \end{itemize}
    \end{itemize}
\end{frame}

%------------------------------------------------
\section{Marco teórico y Estado del arte}

\begin{frame}{Marco teórico: conceptos clave}
    \begin{itemize}
        \item \textbf{IoT y arquitectura en capas}
        \begin{itemize}
            \item Percepción: sensores, actuadores (servomotor, buzzer, LED).
            \item Procesamiento: ESP32 (lógica de decisión).
            \item Red / Aplicación: BLE y WiFi para monitoreo.
        \end{itemize}
        \item \textbf{BLE y RSSI}
        \begin{itemize}
            \item BLE: comunicación de corto alcance y bajo consumo.
            \item RSSI: indicador para estimar proximidad del collar al arenero.
        \end{itemize}
        \item \textbf{Wearables e identificación electrónica}
        \begin{itemize}
            \item Collares electrónicos para animales (peso, autonomía, comodidad).
            \item Tecnologías RFID vs BLE para identificar individuos.
        \end{itemize}
    \end{itemize}
\end{frame}

%------------------------------------------------
\begin{frame}{Estado del arte: wearables y BLE}
    \begin{itemize}
        \item \textbf{Fonseca et al. (2023)}:
        \begin{itemize}
            \item Collar para ovinos con sensores inerciales y de temperatura.
            \item Destacan minimizar peso y número de sensores.
        \end{itemize}
        \item \textbf{Walker et al. (2024)}:
        \begin{itemize}
            \item Dispositivo BLE para proximidad y localización en ovejas.
            \item Analizan relación RSSI--distancia $\Rightarrow$ base para nuestro umbral de apertura.
        \end{itemize}
        \item \textbf{Farine et al. (2024)}:
        \begin{itemize}
            \item Balizas BLE de bajo costo/peso con gran autonomía.
            \item Proveen parámetros para diseñar el collar BLE (intervalo de advertising, tamaño).
        \end{itemize}
    \end{itemize}
\end{frame}

%------------------------------------------------
\begin{frame}{Estado del arte: control selectivo y areneros}
    \begin{itemize}
        \item \textbf{Wild et al. (2023)}:
        \begin{itemize}
            \item Dispositivo selectivo con puertas activadas por tags RFID.
            \item Patrón: identidad en el collar $\Rightarrow$ puerta controla acceso.
        \end{itemize}
        \item \textbf{Brianbojoyou et al. (2023)}:
        \begin{itemize}
            \item Comedero automático con lector RFID y servomotor.
        \end{itemize}
        \item \textbf{Zainal \& Chua (2023); Seema et al. (2025)}:
        \begin{itemize}
            \item Areneros automáticos con ESP32, sensores PIR/ultrasónico y notificaciones.
            \item Validan la viabilidad de integrar microcontroladores, sensores y actuadores en areneros.
        \end{itemize}
        \item \textbf{Gendy et al. (2023)}:
        \begin{itemize}
            \item Comparan BLE vs RFID en entornos IoT.
            \item BLE resulta adecuado cuando se usa ESP32 y se requiere flexibilidad indoor.
        \end{itemize}
    \end{itemize}
\end{frame}

%------------------------------------------------
\section{Metodología y diseño}

\begin{frame}{Metodología general}
    \begin{enumerate}
        \item \textbf{Definición de arquitectura y componentes}
        \begin{itemize}
            \item Selección de ESP32, collar BLE, sensores y actuadores.
        \end{itemize}
        \item \textbf{Desarrollo de firmware}
        \begin{itemize}
            \item Escaneo BLE, filtrado por ID del collar, cálculo de RSSI.
            \item Lógica de decisión para apertura/cierre de compuerta.
        \end{itemize}
        \item \textbf{Integración física y pruebas}
        \begin{itemize}
            \item Montaje en estructura del arenero.
            \item Pruebas funcionales en entorno controlado.
        \end{itemize}
    \end{enumerate}
\end{frame}

%------------------------------------------------
\begin{frame}{Lista de componentes principales}
    \begin{itemize}
        \item \textbf{ESP32 DevKit}: microcontrolador principal, BLE y WiFi integrados.
        \item \textbf{Collar con beacon BLE}: emite ID único del gato autorizado.
        \item \textbf{Sensores}:
        \begin{itemize}
            \item Sensor de presencia (IR / ultrasonido / PIR).
            \item Sensor de fin de carrera en la compuerta.
        \end{itemize}
        \item \textbf{Actuadores}:
        \begin{itemize}
            \item Servomotor para la compuerta.
            \item Buzzer para avisos sonoros.
            \item LED de estado como indicador visual.
        \end{itemize}
        \item Fuente de alimentación, protoboard/PCB y estructura mecánica del arenero.
    \end{itemize}
\end{frame}

%------------------------------------------------
\begin{frame}{Microcontrolador y lógica principal}
    \begin{itemize}
        \item \textbf{ESP32 DevKit} seleccionado por:
        \begin{itemize}
            \item Núcleo de procesamiento suficiente para lógica en tiempo real.
            \item BLE integrado para escanear el collar sin hardware extra.
            \item WiFi opcional para registros y monitoreo futuro.
            \item GPIO suficientes para sensores y actuadores.
        \end{itemize}
        \item Lógica central:
        \begin{itemize}
            \item Escanea beacons BLE, filtra por ID del collar.
            \item Estima proximidad con RSSI.
            \item Si el gato autorizado está cerca y es seguro:
            \begin{itemize}
                \item Abre compuerta (servomotor), enciende LED, opcionalmente activa buzzer.
            \end{itemize}
            \item Cierra compuerta cuando el gato ha entrado/salido.
        \end{itemize}
    \end{itemize}
\end{frame}

%------------------------------------------------
\begin{frame}{Diagrama de bloques del sistema}
    \begin{figure}
        \centering
        \resizebox{0.9\textwidth}{!}{%
        \begin{tikzpicture}[
            font=\scriptsize,
            node distance=0.9cm and 1.4cm,
            block/.style={draw, rounded corners, align=center,
                          minimum width=2.6cm, minimum height=0.9cm},
            smallblock/.style={draw, rounded corners, align=center,
                               minimum width=2.2cm, minimum height=0.8cm},
            sensor/.style={draw, dashed, rounded corners, align=center,
                           minimum width=2.2cm, minimum height=0.8cm},
            line/.style={-Latex}
        ]

        % COLLAR BLE (izquierda)
        \node[smallblock] (collar) {Collar con\\ beacon BLE};

        % ESP32 (centro)
        \node[block, right=of collar] (esp32) {ESP32 DevKit\\
            Escaneo BLE, lógica de decisión\\ y control de actuadores};

        % ACTUADORES (derecha)
        \node[smallblock, right=of esp32, yshift=0.8cm] (servo) {Compuerta\\ (servomotor)};
        \node[smallblock, right=of esp32] (led) {LED de\\ estado};
        \node[smallblock, right=of esp32, yshift=-0.8cm] (buzzer) {Buzzer};

        % SENSORES (abajo)
        \node[sensor, below=1.1cm of esp32, xshift=-1.4cm] (presencia) {Sensor de\\ presencia};
        \node[sensor, below=1.1cm of esp32, xshift=1.4cm] (limit) {Sensor de fin\\ de carrera};

        % WIFI / INTERFAZ (arriba)
        \node[block, above=1.2cm of esp32] (wifi) {Red WiFi /\\ Interfaz de usuario\\ (opcional)};

        % CONEXIONES
        \draw[line] (collar) -- node[above]{ID / RSSI} (esp32);
        \draw[line] (esp32) -- (servo);
        \draw[line] (esp32) -- (led);
        \draw[line] (esp32) -- (buzzer);
        \draw[line] (presencia.north) -- (esp32.south -| presencia.north);
        \draw[line] (limit.north) -- (esp32.south -| limit.north);
        \draw[line] (wifi.south) -- (esp32.north);

        \end{tikzpicture}%
        }
        \caption{Diagrama de bloques del sistema propuesto.}
    \end{figure}
\end{frame}

%------------------------------------------------
\section{Objetivos y alcances}

\begin{frame}{Objetivo general}
    \begin{block}{}
        Diseñar e implementar un prototipo de arenero inteligente con acceso selectivo que,
        mediante un collar basado en BLE y un microcontrolador ESP32, permita el ingreso
        únicamente de un gato autorizado, controlando de manera segura la apertura y cierre de una compuerta.
    \end{block}
\end{frame}

%------------------------------------------------
\begin{frame}{Objetivos específicos}
    \begin{itemize}
        \item Analizar tecnologías de identificación (BLE vs RFID) y justificar la elección de BLE.
        \item Diseñar el collar con beacon BLE (peso, autonomía, fijación).
        \item Desarrollar el firmware del ESP32 para:
        \begin{itemize}
            \item Escaneo BLE y filtrado por ID del collar.
            \item Estimación de proximidad con RSSI y definición de umbrales.
            \item Control de servomotor, buzzer y LED, lectura de sensores.
        \end{itemize}
        \item Integrar electrónica y estructura del arenero con compuerta motorizada.
        \item Realizar pruebas funcionales en entorno controlado.
        \item Documentar arquitectura, decisiones de diseño y limitaciones.
    \end{itemize}
\end{frame}

%------------------------------------------------
\begin{frame}{Alcances y limitaciones}
    \textbf{Alcances}
    \begin{itemize}
        \item Un arenero con una sola compuerta motorizada controlada por ESP32.
        \item Un único gato autorizado (un collar BLE).
        \item Sensores básicos de seguridad y monitoreo mínimo por serie/WiFi.
    \end{itemize}
    \vspace{0.2cm}
    \textbf{Limitaciones}
    \begin{itemize}
        \item No se implementa autolimpieza ni gestión multi-gato en esta fase.
        \item Prototipo de prueba de concepto (sin certificaciones comerciales).
        \item Pruebas en entorno controlado, sin evaluación a largo plazo.
    \end{itemize}
\end{frame}

%------------------------------------------------
\section{Cierre}

\begin{frame}{Conclusiones parciales y siguientes pasos}
    \begin{itemize}
        \item El problema de acceso selectivo a areneros es relevante para el bienestar de gatos en hogares multi-gato.
        \item El análisis del estado del arte respalda el uso de \textbf{collares BLE} y de un \textbf{ESP32} como nodo central.
        \item Se ha definido una arquitectura IoT coherente con los requisitos del curso (sensores, actuadores, comunicación inalámbrica).
        \item \textbf{Siguientes pasos}:
        \begin{itemize}
            \item Implementar y probar el firmware de detección BLE y lógica de compuerta.
            \item Integrar la parte mecánica del arenero y la interfaz de monitoreo.
            \item Medir métricas de desempeño (tasa de detección correcta, falsos positivos, tiempos de apertura).
        \end{itemize}
    \end{itemize}
\end{frame}

%------------------------------------------------

\begin{frame}[allowframebreaks]{Referencias}
\small
\begin{thebibliography}{9}

\bibitem{walker2024ble}
A.~M. Walker, N.~N. Jonsson, A.~Waterhouse \emph{et al.},
``Development of a novel Bluetooth Low Energy device for proximity and location monitoring in grazing sheep,''
\emph{Animal}, vol.~18, p.~101276, 2024.
doi: \url{https://doi.org/10.1016/j.animal.2024.101276}.

\bibitem{fonseca2023wearable}
L.~Fonseca, D.~Corujo, W.~Xavier, and P.~Gonçalves,
``On the Development of a Wearable Animal Monitor,''
\emph{Animals}, vol.~13, no.~1, p.~120, 2023.
doi: \url{https://doi.org/10.3390/ani13010120}.

\bibitem{wild2023selective}
S.~Wild, G.~Alarc\'on-Nieto, M.~Chimento, and L.~M. Aplin,
``Manipulating actions: A selective two-option device for cognitive experiments in wild animals,''
\emph{Journal of Animal Ecology}, vol.~92, no.~8, pp.~1509--1519, 2023.
doi: \url{https://doi.org/10.1111/1365-2656.13756}.

\bibitem{farine2024blebeacons}
D.~R. Farine, J.~Penndorf, S.~Bolcato, B.~Nyaguthii, and L.~M. Aplin,
``Low-cost animal tracking using Bluetooth low energy beacons on a crowd-sourced network,''
\emph{Methods in Ecology and Evolution}, vol.~15, no.~12, pp.~2247--2261, 2024.
doi: \url{https://doi.org/10.1111/2041-210X.14433}.

\bibitem{gendy2023blevsrfid}
M.~E.~G. Gendy, P.~Tham, F.~Harrison, and M.~R. Yuce,
``Comparing Efficiency and Performance of IoT BLE and RFID-Based Systems for Achieving Contact Tracing to Monitor Infection Spread among Hospital and Office Staff,''
\emph{Sensors}, vol.~23, no.~3, p.~1397, 2023.
doi: \url{https://doi.org/10.3390/s23031397}.

\bibitem{brian2023petfeeder}
J.~Brianbojoyou, A.~Ashivin, V.~Abilash, and V.~M. Bhaskaran,
``Automated Pet Feeder with RFID Technology using Design Thinking Approach,''
\emph{International Journal for Research in Applied Science \& Engineering Technology (IJRASET)}, vol.~11, no.~11, pp.~1090--1096, 2023.
doi: \url{https://doi.org/10.22214/ijraset.2023.56668}.

\bibitem{zainal2023litterbox}
M.~H. Zainal and K.~L. Chua,
``Development of Automatic Litter Box Using ESP32,''
\emph{Evolution in Electrical and Electronic Engineering (EEEE)},
vol.~4, no.~2, pp.~529--536, Oct.~2023, Accessed: Nov.~14, 2025.
[Online]. Available:
\url{https://publisher.uthm.edu.my/periodicals/index.php/eeee/article/view/13195}


\bibitem{seema2025littertray}
M.~Seema, M.~Prince, J.~P. P.~M., and A.~Sumithra,
``An IoT-Enabled System for Monitoring and Alerting Cat Litter Tray Cleaning Based on Fill-Level Detection,''
\emph{International Journal of Environmental Sciences}, vol.~11, no.~22s, pp.~5107--5114, 2025.
doi: \url{https://doi.org/10.64252/jezsxz36}.

\end{thebibliography}
\end{frame}

%------------------------------------------------
\begin{frame}
    \centering
    \Huge ¿Preguntas?
\end{frame}

%------------------------------------------------
\end{document}